\chapter{Introduction}

Back in end of {\it 1960’s} when the Internet was a rather small 
network, which was interconnecting major universities, governmental 
and military organizations, very little attention was devoted to 
security. Nowadays, when the Internet has become extremely 
sophisticated in structure, connecting billions of devices ranging 
from small IoT type devices to humongous data-centers, security has 
gained number one priority. In present days, a typical Intranet of 
an organization can include number of geographically separated 
branch-office networks (for example, consider a factory that has 
many SCADA devices and a mission control center that is miles and 
miles away). Since these networks geographically separated, connecting 
them becomes a necessity, and so is the security of these networks. 
This is when the layer-3 virtual private networks ({\it L3-VPN}) and layer-2 
virtual private LAN services ({\it L2-VPLS}) solutions become handy. 
There are, however, other requirements that need to be taken into 
account. Scalability, resilience to various attacks, from man-in-the-middle 
to integrity violation attacks, to rather fundamental attacks on 
asymmetric algorithms (such as RSA, DSA and their elliptic curve 
counterparts, Diffie-Hellman and Eliptic Curve DH, for sampling) 
using, for example, {\it Shor’s quantum computer algorithm} to factorize 
large numbers, and massive brute force attacks on hash algorithms 
should be considered thoroughly. With this in mind, in this work we 
present different security solutions, which can be used to build secure 
L2 and L3 overlay networks. We present the limitations of each solution 
and identify how they can be avoided using various VPLS and L3-VPN.
With this in mind, we start with a background material on cryptography. 
Here we discuss various symmetric and asymmetric encryption algorithms, 
present the definition of hash functions, which considered secure nowadays, 
and discuss several key agreement algorithms. To make the discussion 
complete we present the threat that quantum computers pose for such 
algorithms as RSA and DH, and discuss how post-quantum algorithms such 
as those that are based on lattice can be used as alternative to classical 
algorithms for encryption and signature constructions. Although, not considered 
as part of the present work, future work can be include the performance 
comparison of standardized RSA and DSA algorithms with the performance of 
lattice-based algorithms incorporated into for example Host Identity Protocol 
or event Transport Layer Security protocol. We than move on to discussion of 
TLS, SSL, IPsec, HIP and SSH protocols and how those can be used to achieve 
integrity and confidentiality of data transmitted over insecure channels. 
Afterwards, we move on to the discussion of the results we have obtained 
over the course of several years. Here, we discuss our practical experience 
with scalable Host Identity Protocol based L3-VPN and VPLS network which is 
build using the same protocol. We devote a separate section on 
hardware-accelerated versions of AES and SHA-256 algorithms. We conclude 
the results section with the analysis of the limitations of each solution 
and present the results for the various micro-benchmarking settings. 


\section{Questions}

In this work, we ask several questions. These are not research questions, 
but rather practical questions we try to answer to ourselves in order to 
understand the usability of Python based security solution. Since our work 
focuses on the application of Host Identity Protocol (HIP) in VPN and VPLS 
settings we ask the following questions:

First, {\it what is the performance of the pure Python-based implementation of 
symmetric key encryption and decryption routines as well as hash methods and 
how do they compare to implementation, which uses special CPU AES and SHA-256 
instructions.} Here our focus is on the microbanchmarking of two implementations 
of AES and SHA-256 hashing algorithm, identification of the bottlenecks and 
further recommendations for our prototype implementation of Host Identity Based VPLS and L3-VPN.

Second, {\it what is the scalability of Host Identity Protocol based VPLS and does 
it perform in emulated environment such as Mininet.} Here we seek the answer to 
the question whether the HIP-VPLS is usable in environments close to real-life setups.

Third, {\it what is the performance of Python based HIP-VPLS on real hardware}. By 
doing so we want to find the application niche of our security solution. In addition, 
we discuss the practical configuration of HIP-VPLS using central controller.

The next question {\it we want to answer relates to the deployment of scalable L3-VPN 
based on Host Identity Protocol}. Here we focus on rather different approach of building 
secure networks: We consider L3-VPN where nodes in different branches offices form separate 
broadcast domains, but still can communicate with each other (with assistance of IPv4 
routing protocol). Here, we want to answer how to tackle the scalability issues of 
VPN network by adding hierarchy into the architecture?  


\chapter{Background}

Since we are going to discuss the security protocols in this work, we begin 
this section with the shallow dive into cryptography basics. Here, we discuss 
symmetric and asymmetric cryptography algorithms, to make the description a 
little bit complete we show how RSA algorithm works, discuss the math behind 
Diffie-Hellman (DH) and its Elliptic Curve counterpart. We will also discuss 
the Shor’s algorithm and its quantum computer implementation that theoretically 
can efficiently factorize big number. This algorithm, if powerful enough quantum 
computers will exist in the near future, puts the RSA algorithm - the major 
building block of modern security solutions - at risk of being cracked (once the 
modulus of RSA algorithm factorized into prime components, the private key of RSA 
algorithm can be easily recovered). We will conclude this part of the background 
material with the discussion of post-quantum computer public key encryption 
solution based on lattice (more specifically we will discuss Learning With Errors 
(LWE) algorithm). We believe that, eventually, this type of cryptography will be 
the replacement for traditional RSA and DH algorithms, which rely on the hardness 
of factorization of the big numbers and discrete logarithms. In the epilog of this 
section, we will put few words on how lattice public key cryptography can be used, 
for example, together with Host Identity Protocol.

In the second part of the background material, we will review the basics of the Host 
Identity Protocol, Transport Layer Security Protocol and Secure Shell Protocol, since 
these protocols are the main building blocks of secure tunneling protocols that we
discuss in this work.

We will finalize the discussion of the background material with a short overview of 
various L2, L3 and L4 tunneling solutions, including L2 802.1Q QinQ tunneling, 
L3 Multi-Protocol Label Switching (MPLS) VPLS, L4 TLS and SSH tunneling.


\section{Cryptography basics}

Cryptography comes in many flavors: symmetric key cryptography 
(3DES, AES, Twofish, RC4) which, in turn, can be categorized into 
block cipher and stream cipher and asymmetric key cryptography 
(such as RSA, DSA, ECDSA). There are also key exchange protocols 
such as Diffie-Hellamn and Elyptic Cryptography DH for negotiation 
of common keys over insecure channels. Different algorithms applicable 
in different settings depending on requirements. Typically, as we will 
discuss later, symmetric key cryptography is used to protect 
data-plane traffic in networks, whereas, asymmetric-key cryptography is 
more applicable to the common key negotiation, authentication and 
identification purposes~\cite{Stinson:Cryptography}.

\subsection{Symmetric cryptography}

We start with the symmetric key cryptography. Common key and rather 
trivial operations such as permutations and substitutions are at the 
heart of any symmetric key cryptography algorithm. Although, this type 
of cryptography is efficient because of the usage efficient operations, 
it comes with a limitation though. In symmetric key cryptography, 
both sender and receiver need to share the same key, which complicates 
such important aspects as key distribution and revocation and so alone 
this encryption solutions a very hard to use alone in modern cryptosystems. 
Typically, asymmetric key cryptography such as RSA used to derive session 
keys – TLS, HIP and many other protocols follow this design idea.

Symmetric key cryptography comes in two different flavors: block and stream. 
For example, block cipher (such as AES, 3DES, Twofish~\cite{Stinson:Cryptography}) 
use blocks of data (typically, the size of the block is 128, 160, 256 
bits~\cite{Stinson:Cryptography}), and encrypts or 
decrypts one block at a time. There are different modes of operation, though, 
for block ciphers, examples are counter mode and cipher block chaning. The latter 
one uses so-called initialization vector to add extra randomness into encryption 
process, and encryption of proceeding blocks depends on the output of the previous 
block. Modes of operations are important for security reasons. However, not all 
modes of operations useful and secure. For example, Electronic Code Book (ECB), 
while allow achieving fast processing and parallelization, is considered insecure in 
many settings. In Figure[] we demonstrate AES encryption in CBC mode:

The other type of symmetric key algorithms is stream cipher. Here the encryption 
and decryption performed on separate bits, one bit at a time. CR4 is an example 
of stream cipher. Stream ciphers are extremely important in real-time processing, 
for example, Wi-Fi uses stream ciphers to encrypt the data plane traffic.

\subsection{Asymmetric cryptography}

Asymmetric key cryptography, in its simplest form, is the brilliant in the age 
of computing. Guessing from the name that this type of cryptography uses different 
keys for encryption and decryption does not require deep though. This property makes 
this group of algorithms suitable for various key distribution, revocation and 
signature ideas. 

There is a magnitude of different asymmetric key security algorithms. 
RSA, DSA and its Elliptic curve variant ECDSA are the pillars of modern 
security solutions. But the flexibility of these schemes comes at an extra 
price of CPU cycles. All this makes these solutions inapplicable for securing 
data plane traffic, but only rather to secure control plane. In what follows, 
just to underpin the beauty of the math behind asymmetric key cryptography, 
we provide a description of RSA algorithm.

In RSA cryptosystem, the sender generates a pair of keys as follows: 
First, the sender chooses large enough two prime numbers $p$ and $q$. Next, 
the sender computes $n=pq$ and evaluates Euler’s phi function: $\phi(n)=(p-1)(q-1)$. 
This is the same as the number of numbers coprime to $n$. The sender then selects at 
random encryption exponent $e$ such that $1<e<\phi(n)$ such that $e$ is coprime to $\phi(n)$ 
and computes the decryption exponent d such that $ed \equiv 1\ mod\ \phi(n)$ using modular 
multiplicative inverse.

The public key is then $(n, e)$ and the private key is $(n, d)$. To encrypt the message $m$ 
the sender computes $c= m^e\ mod\ n$. The decryption is similar $m = c^d\ mod\ n$. The beauty is 
in Fermat’s little theorem, which states that $m^{\phi(n)}\ mod\ n \equiv 1\ mod\ n$. 
Now, $ed\ \equiv\ 1\ mod\ \phi(n)$, which means that 
$ed=k\phi(n)+1$, and so $m^{(ed)}\ mod\ n \equiv m^{(k\phi(n)+1)} mod\ n \equiv 1^k m\ mod\ n \equiv m\ mod\ n$. 

In practice, RSA requires random padding to protect against such attacks as chosen ciphertext 
attacks and making two identical plaintext produce various ciphertexts. Padding also ensures that the 
message size is multiple of encryption block-size. In practice, Optimal Asymmetric Encryption Padding (OAEP) 
scheme is used [].

It is good to know that if the message hashed and encrypted with private key, the result is a 
form of digital signature, since the sender cannot later deny that it was involved into encryption 
process. Digital Signature Algorithm (DSA) is another example of asymmetric signature scheme and 
was specifically design for that purpose. Elliptic Curves another version of DSA algorithm.

Frankly speaking, symmetric signature schemes can be also used. For example, one can use 
one-time hash-based signatures to produce the secure digital signatures. Nevertheless, 
the application of these type of signature algorithms is rather impractical and finds 
little application in real-life settings.

\subsection{Cryptographic hash functions}

Mathematically, speaking hash function is special function that given a 
pre-image of an arbitrary size produces an image or hash of a fixed size, 
which is universally unique. Secure hash functions guarantee, to a certain 
degree, that the result of a hash function is irreversible. That it is, it 
should be extremely hard to find a pre-image, or original message, given a 
hash or fingerprint. Secure hash functions should be also collision resistant. 
In other words, it should be extremely hard, if not impossible at all, to 
find two different messages $m$ and $m’$ that will hash to the same value, i.e., 
$hash(m)=hash(m’)$.

Secure hash functions are important in modern cryptography. 
For example, they serve as authentication tokens for transmitted message over the 
wire (useful, for example, in detecting message manipulation during transmission), 
they allow compressing the message before signing it with the digital signature 
algorithm, and they can be used to find the differences between the messages 
efficiently. The application area is of course broader than just these few examples. 

Hash functions come in different flavors, but good ones should be computationally 
efficient and resistant to collisions. Today, hash functions such as MD2, MD4 and MD5 
considered broken, as there are works that showed successful attacks. Briefly speaking, 
researchers successfully found collisions for this hash functions. Therefore, it is 
not recommended to use these hash functions in security applications. A more modern 
family of SHA hash functions also exists. For example, engineers recommend using SHA-256, 
SHA-512 and SHA-3 in modern applications, as no successful attacks were registered for 
these types of hash functions.

Hash functions pave a road for such a notion as authentication tokens when combined with a 
secret key in a special way. Examples are HMAC [], PMAC, CMAC which is based on AES cipher. 
For instance, by sending an HMAC together with the original message one can make sure that 
the message will not be tampered during the transmission. In addition, if the message will 
be altered on the path to recipient, this fact will be flagged immediately during the 
verification process.

Hash functions are also useful in signatures. For example, one-time signature are using 
hash functions to construct an digital signature of a message. An interested reader can 
find more information about hash functions here [].

\subsection{Key exchange protocols}

Key exchange are important in modern systems as they allow negotiation of common 
key over insecure channel. Of course, RSA can be used to deliver a session key
by encrypting it with the recepients public key, but specially crafted key negotiation
algorithms exist in practice. Two bright examples are Diffie-Hellman (DH) and Eliptic Curve 
DH. 


\subsection{Post-quantum Lattice-based cryptography}

Shor's algorithm, implemented on quantum computer makes certain computational 
problems (such as, factoring of large numbers and descrite logarithm problem) 
feasible in polynomial time. This shutters the security of the Internet, and 
so rigorous research was initiated to fill the gap. In what follows we discuss
certain hard mathimatical problem on latticies and show the workings of the 
Learning With Errors (LWE) public key encryption scheme. In fact majority of 
NIST's candidates for post-quantum public key encryption algorithms are based 
on LWE.

A lattice is a mathematical structure which consists of integers in $n$-dimensions 
arranged in a structured lattice-like way. Mathematically, the lattice is defined as follows:
$$\Lambda({\bf B}) = \{{\bf Bx}, {\bf x} \in \mathbb{Z}^n \}$$ where ${\bf B}$ is a matrix of basis vectors
that generate the lattice. We should note that there exist large number of basis vectors, some are {\it good}
some are {\it bad}.

A { \bf closest vector problem (CVP) } on latticies, which is considered NP-hard, is not solvable even on 
quantum computers, can be defined as follows. Given a point $t \in \mathbb{R}^n$ and a lattice
$\Lambda({\bf B})$, the task is to find a closes point ${\bf Bx}$ on lattice: 

$$\min_{\forall {\bf x} \in \mathbb{Z}^n} \|{\bf Bx} - {\bf t}\|$$ 

In practice the above problem is extremely hard to solve which makes lattice-based cryptography 
attractive to cryptographers.

From linear algebra we know that solving equation $\bf{Ax}=\bf{b}$ is simple using
Gaussian elimination. However, if a random noise is added to the equation $$\bf{Ax} + \bf{e} = \bf{b}$$ the 
problem is considered as hard as CVP on lattice. Solving the above problem directly relates to 
solving the CVP problem on lattice if the parameters are selcted carefully.

Given a matrix $\bf{A} \sim U(\mathbb{Z}^{nxm}_q)$, $\bf{s} \sim U(\mathbb{Z}^{n}_q)$ and $\bf{e} \sim D_{\mathbb{Z}^{m}, \sigma}$ 
sampled from discrete (clipped) Gaussian distribution with parameter $\sigma$. We require that, the probability
$P[e < q/4]$ is high (\ie $99.99\%$) to ensure correct decryption of the message and to achieve required 
level of security. This can be achived by setting the parameter for discrete normal distribution carefully, \ie $4\sigma < q/(4m)$.

To ensure that $\bf{re}$ is always less than $q/4$ the values are drawn with resampling
or clipping. We require, further, that the parameter $\sigma = q/(16m)$.

\begin{equation}
    {\bf A}=\begin{bmatrix}
        a_{11}       & a_{12} & a_{13} & \dots & a_{1n} \\
        a_{21}       & a_{22} & a_{23} & \dots & a_{2n} \\
        \hdotsfor{5} \\
        a_{m1}       & a_{m2} & a_{m3} & \dots & a_{mn}
    \end{bmatrix}
\end{equation}

\begin{equation}
    {\bf s}=\begin{bmatrix}
        s_{1} \\
        s_{2} \\
        \dots \\
        s_{n} 
    \end{bmatrix}
\end{equation}

\begin{equation}
    {\bf e}=\begin{bmatrix}
        e_{1} \\
        e_{2} \\
        \dots \\
        e_{n} 
    \end{bmatrix}
\end{equation}

Once the parameters are generated, we can compute ${\bf As}+{\bf e} = {\bf b}$. Then, the public 
key is $({\bf A, b})$ and the private key is ${\bf s}$. Derving ${\bf s}$ from ${\bf b}$ is a hard 
task at hand.

To encrypt the message $\mu \in \{0, 1\}$ choose ${\bf r} \sim U(\{0, 1\}^m)$. Then compute ${\bf u}={\bf rA}$ and 
$v={\bf rb}+\lfloor q/2 \rceil \mu$. The ciphertext is $({\bf u}, v)$. To decrypt the message 
compute $v - \bf{u s}$, if the result is less than $q/4$ output $0$, if the result
is larger than $q/4$ output $1$. 

The major disadvantage of lattice-based cryptography is the size of the keys and actual ciphertext. For example,
security of LWE depends on two parameters $n$ and $q$. By choosing $n=1024$ and $q=2^{32}$, the size of ciphertext
for a message of $256$ bits long (for example, this is the size of the key for AES-256 symmetric algorithm), 
the size of the ciphertext will be $O(n\log q) \approx 4\cdot 1024$ B or roughly $4$ KB. All in all 
the security does not come for free.

\section{Security protocols}

\subsection{Host Identity Protocol (HIP)}

\subsection{Transport Layer Security Protocol (TLS)}

\subsection{Secure Shell Protocol (SSH)}

\section{L2, L3 and L4 tunneling}

\subsection{Virtual Private LAN Services (VPLS) L2VPN solutions}

\subsubsection{QinQ tunneling}

\subsubsection{MPLS tunelling}

\subsection{Virtual Private LAN (L3-VPN) security solutions}

\subsubsection{Multipoint to single point VPN}

\subsubsection{SSH tunneling}

\chapter{Results}

\section{Hardware-enabled symmetric cryptography}

\section{Host Identity Protocol based VPLS}

\section{Scalable multipoint to multipoint VPN using HIP protocol}

\chapter{Conclusions}






